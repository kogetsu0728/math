% https://speakerdeck.com/kogetsu0728/jensennobu-deng-shi

\documentclass[dvipdfmx,aspectratio=169]{beamer}

\usepackage{amsmath}
\usepackage{comment}
\usepackage{url}
\usepackage{pxjahyper}

\usepackage{amsthm}
\theoremstyle{plain}
\newtheorem{thm}{Theorem}

\theoremstyle{definition}
\newtheorem{dfn}{Definition}

\usetheme{Boadilla}

\title{
    Jensenの不等式
}
\author{@kogetsu0728}
\date{\today}

\begin{document}
\maketitle

\section{凹関数}
\begin{frame}{目次}
	\tableofcontents[currentsection]
\end{frame}

\begin{frame}{凹関数の定義}
	上に凸な関数を\textbf{凹関数}(concave function)という\cite{lit:proof}.
	\begin{dfn}
		区間で定義された関数$f:\mathbb{R} \supseteq I \to \mathbb{R}$が凹関数であるとは,
		\[
			\forall x,y \in I, \forall \lambda \in [0,1], f((1-\lambda)x + \lambda y) \ge (1-\lambda)f(x) + \lambda f(y).
		\]
		を満たすことをいう.
	\end{dfn}

	また,次の必要十分条件が知られている.証明は省略する.
	\begin{thm}
		$f$が二階微分可能ならば,
		\[
			\text{$f$が凹関数} \iff f''(x) \le 0.
		\]
	\end{thm}
\end{frame}

\begin{frame}{凹関数の判定(定義から)}
	\begin{thm}
		$log_2 x$は凹関数である.
	\end{thm}

	\begin{proof}
		$x,y > 0$,$\lambda \in [0, 1]$とする.

		\textbf{重み付き相加相乗平均の不等式}(weighted AM-GM inequality)から,
		\[
			(1-\lambda)x + \lambda y \ge x^{1-\lambda} y^\lambda.
		\]
		両辺の対数を取ると,
		\[
			\log_2 ((1-\lambda)x + \lambda y) \ge \log_2 (x^{1-\lambda} y^\lambda) = (1-\lambda)\log_2 x + \lambda \log_2 y.
		\]
	\end{proof}
\end{frame}

\begin{frame}{凹関数の判定(二階微分から)}
	\begin{thm}
		$log_2 x$は凹関数である.
	\end{thm}

	\begin{proof}
		\[
			(\log_2 x)' = \frac{1}{(\log 2)x}, \qquad
			(\log_2 x)'' = -\frac{1}{(\log 2)x^2} \le 0.
		\]
	\end{proof}
\end{frame}

\section{Jensenの不等式}
\begin{frame}{目次}
	\tableofcontents[currentsection]
\end{frame}

\begin{frame}{Jensenの不等式}
	以下の不等式を\textbf{Jensenの不等式}(Jensen's inequality)という\cite{lit:kanketsu}.

	\begin{thm}
		$f:\mathbb{R} \supseteq I \to \mathbb{R}$を凹関数とする.
		また,$p_{1}, p_{2}, \ldots, p_{n}$を$p_{1} + p_{2} + \cdots + p_{n} = 1$を満たす非負実数とし,
		$x_{1}, x_{2}, \ldots, x_{n} \in I$とすると,
		\[
			\sum_{i=1}^{n} p_{i}f(x_{i}) \le f(\sum_{i=1}^{n} p_{i}x_{i}).
		\]
	\end{thm}

	数学的帰納法で証明できる.
	特に,$n=2$のときは凹関数の定義より明らか.

	\vskip\baselineskip
	符号長を表す式は$log_2 x$など凹関数が多く,上界を求めるのにJensenの不等式が役立つ.
\end{frame}

\begin{frame}{不等式の証明}
	\textbf{平均情報量}(entropy)に関する以下の不等式をJensenの不等式を用いて証明する.
	\begin{thm}
		$X$を$M$個の事象の生起確率$p_{1}, p_{2}, \ldots, p_{M}$からなる確率事象系とする.
		このとき,平均情報量$H(X)$について次の不等式が成り立つ.
		\[
			H(X) \le \log_2 M.
		\]
	\end{thm}

	\begin{proof}
		$\log_2 x$は凹関数だから,Jensenの不等式より,
		\[
			H(X) = \sum_{i=1}^M p_{i} \log_2 \frac{1}{p_{i}} \le \log_2 (\sum_{i=1}^M p_{i} \frac{1}{p_{i}}) = \log_2 M.
		\]
	\end{proof}
\end{frame}

\section*{参考文献}
\begin{frame}{参考文献}
	\begin{thebibliography}{9}
		\renewcommand{\baselinestretch}{1.0}
		\small
		\beamertemplatetextbibitems
		\bibitem{lit:proof}
		高校数学の美しい物語,
		``上に凸,下に凸な関数と二階微分'',
		https://manabitimes.jp/math/927
		(参照: 2025-06-26).

		\bibitem{lit:kanketsu}
		定兼 邦彦,
		``簡潔データ構造'',
		共立出版, 2018.

	\end{thebibliography}
\end{frame}

\end{document}

