\documentclass[dvipdfmx,aspectratio=169]{beamer}

\usepackage{amsmath}
\usepackage{comment}
\usepackage{url}
\usepackage{pxjahyper}

\usepackage{amsthm}
\theoremstyle{plain}
\newtheorem{thm}{Theorem}

\theoremstyle{definition}
\newtheorem{dfn}{Definition}

\usetheme{Boadilla}

\title{
    Jensenの不等式
}
\author{@kogetsu0728}
\date{\today}

\begin{document}
\maketitle

\section{凹関数}
\begin{frame}{目次}
	\tableofcontents[currentsection]
\end{frame}

\begin{frame}{凹関数の定義}
    上に凸な関数を\textbf{凹関数}(concave function)という\cite{lit:proof}.
    \begin{dfn}
        $f(x)$を実数上の関数とする.また,$x,y$を実数,$\lambda$を$0 \le \lambda \le 1$を満たす実数とする.
        \[
            f((1-\lambda)x + \lambda y) \ge (1-\lambda)f(x) + \lambda f(y)
        \]
        を満たすとき,またその時に限り$f(x)$を凹関数という.
    \end{dfn}

    以下の定理が知られている.証明は省略する.
    \begin{thm}
        $f(x)$を実数上の関数とする. $f(x)$が二階微分可能ならば,
        \[
            \text{$f(x)$は凹関数} \iff f''(x) \le 0.
        \]
    \end{thm}
\end{frame}

\begin{frame}{凹関数の判定}
    二階微分を用いて,凹関数の判定を行う.

    \begin{thm}
        $log_2 x$は凹関数である.
    \end{thm}

    \begin{proof}
        \[
            (\log_2 x)' = \frac{1}{x \log 2}, \qquad
            (\log_2 x)'' = -\frac{1}{x^2 \log 2} \le 0.
        \]
    \end{proof}
\end{frame}

\section{Jensenの不等式}
\begin{frame}{目次}
	\tableofcontents[currentsection]
\end{frame}

\begin{frame}{Jensenの不等式}
    以下の不等式を\textbf{Jensenの不等式}という\cite{lit:kanketsu}.

    \begin{thm}
        $f(x)$を凹関数とする.
        また,$p_{1}, p_{2}, \ldots, p_{n}$を$p_{1} + p_{2} + \cdots + p_{n} = 1$である非負の実数とし,
        $x_{1}, x_{2}, \ldots, x_{n}$を実数とすると,
        \[
            \sum_{i=1}^{n} p_{i}f(x_{i}) \le f(\sum_{i=1}^{n} p_{i}x_{i}).
        \]
    \end{thm}

    数学的帰納法で証明できる.
    $n=2$のときは凹関数の定義より明らか.

	\vskip\baselineskip
    符号長を表す式は$log_2 x$など凹関数が多く,上界を求めるのにJensenの不等式が役立つ.
\end{frame}

\begin{frame}{不等式の証明}
    エントロピーに関する以下の不等式をJensenの不等式を用いて証明する.
    \begin{thm}
        \[
        H(A) \le \log_2 M.
        \]
    \end{thm}

    \begin{proof}
        $\log_2 x$は凹関数だから,Jensenの不等式より,
        \[
            H(A) = \sum_{i=1}^M p_{i} \log_2 \frac{1}{p_{i}} \le \log_2 (\sum_{i=1}^M p_{i} \frac{1}{p_{i}}) = \log_2 M.
        \]
    \end{proof}
\end{frame}

\section*{参考文献}
\begin{frame}{参考文献}
\begin{thebibliography}{9}
\renewcommand{\baselinestretch}{1.0}
\small
\beamertemplatetextbibitems
\bibitem{lit:proof}
    数学の景色,
    ``凸関数と凸不等式(イェンセンの不等式)についてかなり詳しく'',
    https://mathlandscape.com/convex-func/,
    (参照: 2025-06-26).

\bibitem{lit:kanketsu}
    定兼 邦彦,
    ``簡潔データ構造'',
    共立出版, 2018.

\end{thebibliography}
\end{frame}

\end{document}

